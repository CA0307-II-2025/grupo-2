\documentclass[12pt, a4paper]{article}
% Paquetes
\usepackage[spanish]{babel}
\usepackage[utf8]{inputenc}
\usepackage{graphicx}
\usepackage{geometry}
\geometry{left=3cm,right=3cm,top=3cm,bottom=3cm}
\usepackage{setspace}
\setstretch{1.5}
\usepackage{apacite}
\bibliographystyle{apacite}

\begin{document}

% Portada
\begin{titlepage}
    \centering

    % Logo opcional
    \includegraphics[width=0.3\textwidth]{LogoUCR.png}\par\vspace{1cm}

    {\scshape\LARGE Universidad De Costa Rica \par}
    \vspace{1.5cm}

    {\Huge\bfseries Proyecto Estadística II\par}
    \vspace{1.5cm}

    {\large \bfseries Autores:\par}
    {\Large Dixon Montero Hernández - B99109 \par}
    {\Large Jose Andrey Prado Rojas - C36174 \par}
    {\Large Joseph Romero Chinchilla - C37006\par}
    \vspace{0.5cm}
    {\large Facultad de Ciencias Básicas, Universidad De Costa Rica\par}
    {\large CA0307: Estadística Actuarial II}
    \vspace{2cm}

    {\large\bfseries Profesor:\par}
    {\large Dr. Maikol Solís Chacón}

    \vfill

    {\large \today\par}
\end{titlepage}

\section*{Introducción}

El estudio de fenómenos extremos vinculados a catástrofes naturales, tales como inundaciones o terremotos, ha experimentado un avance significativo en los últimos años mediante la adopción de metodologías probabilísticas multivariadas. Estas metodologías permiten representar de manera más precisa la compleja interdependencia entre variables esenciales, ofreciendo un marco analítico robusto frente a la incertidumbre inherente a los eventos extremos. En este contexto, las cópulas se han consolidado como una herramienta estadística fundamental, ya que posibilitan construir distribuciones conjuntas a partir de distribuciones marginales, separando el modelado de cada variable individual de la estructura de dependencia que las vincula. Esta flexibilidad resulta particularmente relevante en el análisis de fenómenos naturales extremos, dado que las variables involucradas rara vez presentan distribuciones normales y suelen estar fuertemente correlacionadas \cite{DelfinerGutierrez2025,PerezGarcia2004}.

Paralelamente, la teoría de valores extremos (Extreme Value Theory, EVT) se ha establecido como un marco esencial para modelar y predecir eventos raros y catastróficos, proporcionando herramientas matemáticas capaces de estimar las colas de distribución de dichos fenómenos \cite{Siddiqui2022,DelfinerGutierrez2025}. La integración de EVT con cópulas permite capturar simultáneamente la dependencia entre múltiples variables extremas, lo que resulta crucial para obtener estimaciones precisas de riesgos multivariados, como la combinación de intensidad y duración de lluvias extremas o la simultaneidad de aceleraciones sísmicas en distintos modos estructurales.

En el ámbito hidrológico, la aplicación de cópulas ha favorecido la estimación de riesgos de inundaciones, evidenciada en investigaciones que utilizan modelos de vine copula para evaluar la probabilidad conjunta de que la duración, el volumen y el pico de una crecida excedan determinados umbrales \cite{DelfinerGutierrez2025}. Este enfoque integral permite medir con mayor exactitud las probabilidades de eventos multivariados, proporcionando información clave para el diseño de sistemas de alerta temprana y estrategias de mitigación. Aunque la probabilidad combinada de ocurrencia de eventos extremos suele ser baja, su correcta identificación y modelado contribuye a reducir la incertidumbre en la gestión del riesgo y optimizar la toma de decisiones \cite{PerezGarcia2004}.

De manera análoga, en ingeniería sísmica, las cópulas han demostrado su utilidad al evaluar riesgos desde una perspectiva multivariada. Un caso ilustrativo corresponde a un estudio probabilístico de un rascacielos de acero en la Ciudad de México, en el que se consideraron como variables de entrada las aceleraciones espectrales asociadas a los primeros modos de vibración y como variable de salida la máxima deformación entre los pisos. La modelización de la relación entre estas variables mediante cópulas permitió obtener tasas de excedencia conjuntas más precisas que las ofrecidas por enfoques univariados tradicionales, los cuales tienden a sobreestimar los riesgos \cite{Siddiqui2022}. Este ejemplo evidencia cómo la incorporación de dependencias estadísticas mejora la exactitud de las estimaciones y fortalece la evaluación de la seguridad estructural ante eventos sísmicos.

En el ámbito asegurador y de gestión de riesgos financieros, la modelización probabilística de eventos extremos permite cuantificar pérdidas potenciales derivadas de catástrofes naturales y establecer primas adecuadas para la transferencia de riesgo. La combinación de cópulas con análisis de valor en riesgo (VaR) y técnicas de simulación proporciona herramientas robustas para evaluar escenarios de pérdidas catastróficas y diseñar estrategias de mitigación frente a incertidumbres significativas \cite{PerezGarcia2004}.

En este trabajo, se pretende aplicar estas metodologías al estudio de desastres naturales en Costa Rica, considerando fenómenos como inundaciones, deslizamientos y eventos sísmicos, y generando un marco probabilístico que permita evaluar y gestionar el riesgo de manera más precisa. Este enfoque busca adaptarse a las particularidades geográficas, hidrológicas y sísmicas del país, ofreciendo información valiosa para la prevención, planificación y toma de decisiones en contextos de alta incertidumbre.

En conjunto, estos avances evidencian que la aplicación de cópulas en el análisis de desastres naturales proporciona una visión integral y versátil. Al capturar las relaciones entre variables clave y estimar la probabilidad real de eventos extremos multivariados, las cópulas se consolidan como un método esencial para diseñar estrategias de prevención, optimizar el diseño estructural y administrar el riesgo en escenarios complejos e inciertos \cite{DelfinerGutierrez2025,Siddiqui2022,PerezGarcia2004}.

\section*{Marco Teórico}

\subsection*{Conceptualización de los desastres naturales}
Los \textit{desastres naturales} son fenómenos que, al interactuar con condiciones de vulnerabilidad social y económica, producen efectos adversos en la población, la infraestructura y el entorno \cite{Paniagua1995}. Estos eventos no deben entenderse únicamente como expresiones de la naturaleza, sino también como hechos sociales, en tanto ponen de manifiesto desigualdades, falta de planificación y limitaciones en la capacidad de respuesta de las comunidades \cite{PerezMallaina2005}.  

En el caso de América Central y particularmente Costa Rica, estudios recientes han documentado la recurrencia de eventos sísmicos, inundaciones y erupciones volcánicas, señalando que estos han generado tanto desplazamientos poblacionales como impactos fiscales de consideración \cite{CentenoMorales2017,OrozcoMontoya2022}.  

\subsection*{Impacto económico de los desastres naturales}
Las catástrofes naturales tienen efectos inmediatos y de largo plazo en la economía. Los costos directos incluyen daños a infraestructura, viviendas y sistemas de transporte, mientras que los indirectos abarcan la pérdida de productividad, la interrupción de cadenas de suministro y el aumento en gastos sociales y de salud \cite{Paniagua1995}. En contextos de alta vulnerabilidad, estos gastos pueden comprometer seriamente la sostenibilidad financiera de los Estados.  

De acuerdo con \cite{PerezMallaina2005}, los desastres también pueden ser entendidos como \textit{instrumentos de observación social}, en tanto revelan dinámicas de poder, inequidades y limitaciones institucionales, lo que ayuda a comprender por qué ciertos sectores son más afectados que otros.  

\subsection*{Modelación probabilística del riesgo catastrófico}
El análisis del riesgo asociado a los desastres naturales se ha beneficiado de herramientas estadísticas avanzadas. Entre ellas destacan las \textit{cópulas}, que permiten modelar dependencias entre variables extremas. Estos modelos son especialmente útiles cuando se analizan fenómenos multivariados como lluvias intensas, caudales fluviales o movimientos sísmicos, donde la correlación tradicional no resulta suficiente para capturar la complejidad de la relación \cite{patton2012review, krupskii2013factor}.  

En el ámbito regional, \cite{RiveraVargas2022} aplicaron cópulas en un análisis probabilístico de peligro sísmico, mostrando su utilidad para evaluar riesgos conjuntos y calcular pérdidas esperadas en escenarios de alta dependencia. Asimismo, \cite{Chand2024} demostraron que el uso de cópulas en la evaluación de riesgo de inundaciones permite una estimación más precisa de los gastos potenciales en infraestructura y mitigación.  

\subsection*{Teoría del valor extremo y su aplicación en catástrofes}
La \textit{teoría del valor extremo} (EVT, por sus siglas en inglés) constituye otra herramienta fundamental para la modelación de desastres. Esta teoría se centra en los eventos poco frecuentes pero de gran magnitud, precisamente los que generan los mayores impactos económicos \cite{Siddiqui2022}.  

Estudios pioneros aplicaron EVT al ajuste y modelación de catástrofes con el fin de estimar pérdidas máximas y diseñar mecanismos de financiamiento adecuados \cite{PerezGarcia2004}. Investigaciones más recientes amplían esta perspectiva al contexto financiero y asegurador, subrayando su valor para la gestión del riesgo catastrófico \cite{DelfinerGutierrez2025}.  

La importancia de EVT radica en que permite calcular métricas como el \textit{Value at Risk (VaR)} y el \textit{Expected Shortfall (ES)} en escenarios extremos, herramientas que orientan tanto a gobiernos como a aseguradoras en la planificación del gasto post-desastre.  

\subsection*{Gastos gubernamentales y resiliencia económica}
Los gastos tras un desastre natural suelen dividirse en:  
\begin{itemize}
    \item \textbf{Atención inmediata}: rescate, albergues temporales, distribución de víveres.  
    \item \textbf{Reconstrucción}: reparación de infraestructura, vivienda, sistemas de transporte y servicios públicos.  
    \item \textbf{Recuperación económica}: subsidios, reactivación de actividades productivas, apoyo a sectores estratégicos.  
\end{itemize}  

La resiliencia económica de un país frente a desastres está determinada por la existencia de fondos de contingencia, seguros catastróficos y mecanismos de financiamiento internacional. La ausencia de estos mecanismos genera altos niveles de endeudamiento y limita la inversión futura en desarrollo \cite{CentenoMorales2017,OrozcoMontoya2022}.  

\subsection*{Relevancia para la investigación actual}
El estudio de los desastres naturales y los gastos que estos implican requiere una aproximación interdisciplinaria que combine historia, ciencias sociales, economía y estadística avanzada. Herramientas como las \textit{cópulas} y la \textit{teoría del valor extremo} no solo permiten estimar con mayor precisión la probabilidad de eventos extremos, sino también anticipar los costos asociados, mejorando la planificación financiera y la política pública.  

Así, este trabajo plantea la necesidad de integrar perspectivas tradicionales sobre vulnerabilidad y desigualdad con modelos probabilísticos modernos, con el fin de diseñar estrategias de mitigación y financiamiento que reduzcan los impactos económicos y sociales de los desastres en la región.  



\nocite{*}% Mostrar todas
\bibliography{referencias}
\end{document}