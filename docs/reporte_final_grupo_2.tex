\documentclass[12pt, a4paper]{article}
% Paquetes
\usepackage[spanish]{babel}
\usepackage[utf8]{inputenc}
\usepackage{graphicx}
\usepackage{geometry}
\geometry{left=3cm,right=3cm,top=3cm,bottom=3cm}
\usepackage{setspace}
\setstretch{1.5}
\usepackage{apacite}
\bibliographystyle{apacite}

\begin{document}

% Portada
\begin{titlepage}
    \centering

    % Logo opcional
    \includegraphics[width=0.3\textwidth]{LogoUCR.png}\par\vspace{1cm}

    {\scshape\LARGE Universidad De Costa Rica \par}
    \vspace{1.5cm}

    {\Huge\bfseries Proyecto Estadística II\par}
    \vspace{1.5cm}

    {\large \bfseries Autores:\par}
    {\Large Dixon Montero Hernández - B99109 \par}
    {\Large Jose Andrey Prado Rojas - C36174 \par}
    {\Large Joseph Romero Chinchilla - C37006\par}
    {\Large Holmar Adrian Rivera Castellon - B86564\par}
    \vspace{0.5cm}
    {\large Facultad de Ciencias Básicas, Universidad De Costa Rica\par}
    {\large CA0307: Estadística Actuarial II}
    \vspace{1cm}

    {\large\bfseries Profesor:\par}
    {\large Dr. Maikol Solís Chacón}

    \vfill

    {\large \today\par}
\end{titlepage}

\section*{Introducción}

El estudio de fenómenos extremos vinculados a catástrofes naturales, tales como inundaciones o terremotos, ha experimentado un avance significativo en los últimos años mediante la adopción de metodologías probabilísticas multivariadas. Estas metodologías permiten representar de manera más precisa la compleja interdependencia entre variables esenciales, ofreciendo un marco analítico robusto frente a la incertidumbre inherente a los eventos extremos. En este contexto, las cópulas se han consolidado como una herramienta estadística fundamental, ya que posibilitan construir distribuciones conjuntas a partir de distribuciones marginales, separando el modelado de cada variable individual de la estructura de dependencia que las vincula. Esta flexibilidad resulta particularmente relevante en el análisis de fenómenos naturales extremos, dado que las variables involucradas rara vez presentan distribuciones normales y suelen estar fuertemente correlacionadas \cite{DelfinerGutierrez2025,PerezGarcia2004}.

Paralelamente, la teoría de valores extremos (Extreme Value Theory, EVT) se ha establecido como un marco esencial para modelar y predecir eventos raros y catastróficos, proporcionando herramientas matemáticas capaces de estimar las colas de distribución de dichos fenómenos \cite{Siddiqui2022,DelfinerGutierrez2025}. La integración de EVT con cópulas permite capturar simultáneamente la dependencia entre múltiples variables extremas, lo que resulta crucial para obtener estimaciones precisas de riesgos multivariados, como la combinación de intensidad y duración de lluvias extremas o la simultaneidad de aceleraciones sísmicas en distintos modos estructurales.

En el ámbito hidrológico, la aplicación de cópulas ha favorecido la estimación de riesgos de inundaciones, evidenciada en investigaciones que utilizan modelos de vine copula para evaluar la probabilidad conjunta de que la duración, el volumen y el pico de una crecida excedan determinados umbrales \cite{DelfinerGutierrez2025}. Este enfoque integral permite medir con mayor exactitud las probabilidades de eventos multivariados, proporcionando información clave para el diseño de sistemas de alerta temprana y estrategias de mitigación. Aunque la probabilidad combinada de ocurrencia de eventos extremos suele ser baja, su correcta identificación y modelado contribuye a reducir la incertidumbre en la gestión del riesgo y optimizar la toma de decisiones \cite{PerezGarcia2004}.

De manera análoga, en ingeniería sísmica, las cópulas han demostrado su utilidad al evaluar riesgos desde una perspectiva multivariada. Un caso ilustrativo corresponde a un estudio probabilístico de un rascacielos de acero en la Ciudad de México, en el que se consideraron como variables de entrada las aceleraciones espectrales asociadas a los primeros modos de vibración y como variable de salida la máxima deformación entre los pisos. La modelización de la relación entre estas variables mediante cópulas permitió obtener tasas de excedencia conjuntas más precisas que las ofrecidas por enfoques univariados tradicionales, los cuales tienden a sobreestimar los riesgos \cite{Siddiqui2022}. Este ejemplo evidencia cómo la incorporación de dependencias estadísticas mejora la exactitud de las estimaciones y fortalece la evaluación de la seguridad estructural ante eventos sísmicos.

En el ámbito asegurador y de gestión de riesgos financieros, la modelización probabilística de eventos extremos permite cuantificar pérdidas potenciales derivadas de catástrofes naturales y establecer primas adecuadas para la transferencia de riesgo. La combinación de cópulas con análisis de valor en riesgo (VaR) y técnicas de simulación proporciona herramientas robustas para evaluar escenarios de pérdidas catastróficas y diseñar estrategias de mitigación frente a incertidumbres significativas \cite{PerezGarcia2004}.

En este trabajo, se pretende aplicar estas metodologías al estudio de desastres naturales en Costa Rica, considerando fenómenos como inundaciones, deslizamientos y eventos sísmicos, y generando un marco probabilístico que permita evaluar y gestionar el riesgo de manera más precisa. Este enfoque busca adaptarse a las particularidades geográficas, hidrológicas y sísmicas del país, ofreciendo información valiosa para la prevención, planificación y toma de decisiones en contextos de alta incertidumbre.

En conjunto, estos avances evidencian que la aplicación de cópulas en el análisis de desastres naturales proporciona una visión integral y versátil. Al capturar las relaciones entre variables clave y estimar la probabilidad real de eventos extremos multivariados, las cópulas se consolidan como un método esencial para diseñar estrategias de prevención, optimizar el diseño estructural y administrar el riesgo en escenarios complejos e inciertos \cite{DelfinerGutierrez2025,Siddiqui2022,PerezGarcia2004}.

\section*{Marco Teórico}

Los \textit{desastres naturales} son fenómenos que, al interactuar con condiciones de vulnerabilidad social y económica, producen efectos adversos en la población, la infraestructura y el entorno \cite{Paniagua1995}. Estos eventos no deben entenderse únicamente como expresiones de la naturaleza, sino también como hechos sociales, en tanto ponen de manifiesto desigualdades, falta de planificación y limitaciones en la capacidad de respuesta de las comunidades \cite{PerezMallaina2005}. En el caso de América Central y particularmente Costa Rica, estudios recientes han documentado la recurrencia de eventos sísmicos, inundaciones y erupciones volcánicas, señalando que estos han generado tanto desplazamientos poblacionales como impactos fiscales de consideración \cite{CentenoMorales2017,OrozcoMontoya2022}.

La gestión del riesgo financiero asociado a desastres naturales se ha abordado en el plano internacional con el Marco de Sendai para la Reducción del Riesgo de Desastres, cuyo propósito central es minimizar las pérdidas económicas y materiales derivadas de fenómenos naturales \cite{undrr2015}. Este marco global constituye una referencia fundamental, pues no se limita a proponer estrategias de respuesta, sino que impulsa la incorporación del riesgo en la planificación del desarrollo, tanto económico como social. En consecuencia, se reconoce que la reducción de pérdidas es también asunto de sostenibilidad fiscal y estabilidad macroeconómica. En el contexto de Costa Rica, la Comisión Nacional de Prevención de Riesgos y Atención de Emergencias ha planteado el Plan Nacional de Gestión del Riesgo 2021–2025, el cual establece mecanismos que buscan cuantificar y enfrentar las consecuencias económicas de los desastres \cite{cne2022}. De esta manera, se observa una transición de un modelo centrado en atender emergencias después de ocurridas, hacia un enfoque preventivo.

Las catástrofes naturales tienen efectos inmediatos y de largo plazo en la economía. Los costos directos incluyen daños a infraestructura, viviendas y sistemas de transporte, mientras que los indirectos abarcan la pérdida de productividad, la interrupción de cadenas de suministro y el aumento en gastos sociales y de salud \cite{Paniagua1995}. En contextos de alta vulnerabilidad, estos gastos pueden comprometer seriamente la sostenibilidad financiera de los Estados. En América Latina y el Caribe, organismos internacionales como el Banco Mundial han advertido sobre la brecha existente entre los costos económicos de los desastres y la capacidad de respuesta fiscal, lo cual genera presión sobre los presupuestos públicos y limita el margen de acción para la inversión en desarrollo \cite[p. 23]{bancamundial2021}. Consecuentemente, la Comisión Económica para América Latina y el Caribe documentó que entre 1990 y 2012 las pérdidas económicas anuales de la región superaron en promedio el 1\% del PIB, cifra que ilustra no solo la magnitud del impacto financiero, sino también la vulnerabilidad estructural de los países ante eventos extremos \cite[p. 45]{cepal2014}.

Estos fenómenos no solo afectan a las comunidades vulnerables, sino que también generan perturbaciones en sectores estratégicos como la agricultura, la energía y el turismo. Además, las proyecciones climáticas resaltan la necesidad de herramientas predictivas más robustas y de esquemas de financiamiento adaptativo que permitan a las instituciones responder de forma oportuna. En este sentido, el análisis del riesgo asociado a los desastres naturales se ha beneficiado de herramientas estadísticas avanzadas. Entre ellas destacan las \textit{cópulas}, que permiten modelar dependencias entre variables extremas, lo cual resulta especialmente útil cuando se analizan fenómenos multivariados como lluvias intensas, caudales fluviales o movimientos sísmicos, donde la correlación tradicional no resulta suficiente para capturar la complejidad de la relación \cite{patton2012review, krupskii2013factor}. En el ámbito regional, \citeA{RiveraVargas2022} aplicaron cópulas en un análisis probabilístico de peligro sísmico, mostrando su utilidad para evaluar riesgos conjuntos y calcular pérdidas esperadas en escenarios de alta dependencia. Asimismo, \citeA{Chand2024} demostraron que el uso de cópulas en la evaluación de riesgo de inundaciones permite una estimación más precisa de los gastos potenciales en infraestructura y mitigación.

La \textit{teoría del valor extremo} (EVT, por sus siglas en inglés) constituye otra herramienta fundamental para la modelación de desastres, ya que se centra en los eventos poco frecuentes pero de gran magnitud, precisamente los que generan los mayores impactos económicos \cite{Siddiqui2022}. Estudios pioneros aplicaron EVT al ajuste y modelación de catástrofes con el fin de estimar pérdidas máximas y diseñar mecanismos de financiamiento adecuados \cite{PerezGarcia2004}. Investigaciones más recientes amplían esta perspectiva al contexto financiero y asegurador, subrayando su valor para la gestión del riesgo catastrófico \cite{DelfinerGutierrez2025}. La importancia de EVT radica en que permite calcular métricas como el \textit{Value at Risk (VaR)} y el \textit{Expected Shortfall (ES)} en escenarios extremos, herramientas que orientan tanto a gobiernos como a aseguradoras en la planificación del gasto post-desastre.

Los gastos tras un desastre natural suelen dividirse en atención inmediata, que incluye rescate, albergues temporales y distribución de víveres; reconstrucción, enfocada en la reparación de infraestructura, vivienda, sistemas de transporte y servicios públicos; y recuperación económica, mediante subsidios, reactivación de actividades productivas y apoyo a sectores estratégicos. La resiliencia económica de un país frente a desastres está determinada por la existencia de fondos de contingencia, seguros catastróficos y mecanismos de financiamiento internacional. La ausencia de estos mecanismos genera altos niveles de endeudamiento y limita la inversión futura en desarrollo \cite{CentenoMorales2017,OrozcoMontoya2022}.

En este contexto, la articulación entre el sector público, el sector privado y la academia desempeña un papel crucial en la creación de mecanismos financieros innovadores, como los seguros paramétricos y los bonos catastróficos. Estos instrumentos no solo distribuyen el riesgo, sino que también fortalecen la resiliencia financiera frente a eventos de gran magnitud \cite{quesada2020}. Sin embargo, aún persisten desafíos importantes en la coordinación de actores y en la generación de confianza, lo que limita la efectividad de estas herramientas. Además, el involucramiento activo del sector privado sigue siendo una tarea pendiente, especialmente en países en vías de desarrollo donde las estructuras de mercado no siempre favorecen la adopción de soluciones financieras de carácter preventivo.

En síntesis, el estudio de los desastres naturales y los gastos que estos implican requiere una aproximación interdisciplinaria que combine historia, ciencias sociales, economía, estadística avanzada y gestión pública. Este trabajo plantea la necesidad de integrar perspectivas tradicionales sobre vulnerabilidad y desigualdad con marcos de gobernanza internacional y modelos probabilísticos modernos, con el fin de diseñar estrategias de mitigación y financiamiento que reduzcan los impactos económicos y sociales de los desastres en la región y contribuyan a garantizar la estabilidad económica y social a largo plazo.

\newpage
\section*{Anexos}
Link del Repositorio de GitHub : \url{https://github.com/CA0307-II-2025/grupo-2}

Imágenes del Dashboard (Avance) :

\begin{figure}[htbp]
    \centering
    \includegraphics[width=0.9\textwidth]{fig/avance_dash_1/1.png}
\end{figure}
    \vspace{0.5cm}
\begin{figure}[htbp]
    \centering
    \includegraphics[width=0.9\textwidth]{fig/avance_dash_1/1.png}
\end{figure}
    \vspace{0.5cm}
\begin{figure}[htbp]
    \centering
    \includegraphics[width=0.9\textwidth]{fig/avance_dash_1/1.png}
\end{figure}
    \vspace{0.5cm}
\begin{figure}[htbp]
    \centering
    \includegraphics[width=0.9\textwidth]{fig/avance_dash_1/1.png}
\end{figure}
    \vspace{0.5cm}
\begin{figure}[htbp]
    \centering
    \includegraphics[width=0.9\textwidth]{fig/avance_dash_1/1.png}
\end{figure}

\newpage
\nocite{*}% Mostrar todas
\bibliography{referencias}
\end{document}
