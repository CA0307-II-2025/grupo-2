\documentclass[12pt, a4paper]{article}
% Paquetes
\usepackage[spanish]{babel}
\usepackage[utf8]{inputenc}
\usepackage{graphicx}
\usepackage{geometry}
\geometry{left=3cm,right=3cm,top=3cm,bottom=3cm}
\usepackage{setspace}
\setstretch{1.5}
\usepackage{apacite}
\bibliographystyle{apacite}

\begin{document}

% Portada
\begin{titlepage}
    \centering

    % Logo opcional
    \includegraphics[width=0.3\textwidth]{LogoUCR.png}\par\vspace{1cm}

    {\scshape\LARGE Universidad De Costa Rica \par}
    \vspace{1.5cm}

    {\Huge\bfseries Proyecto Estadística II\par}
    \vspace{1.5cm}

    {\large \bfseries Autores:\par}
    {\Large Dixon Montero Hernández - B99109 \par}
    {\Large Jose Andrey Prado Rojas - C36174 \par}
    {\Large Joseph Romero Chinchilla - C37006\par}
    {\Large Holmar Adrian Rivera Castellon - B86564\par}
    \vspace{0.5cm}
    {\large Facultad de Ciencias Básicas, Universidad De Costa Rica\par}
    {\large CA0307: Estadística Actuarial II}
    \vspace{1cm}

    {\large\bfseries Profesor:\par}
    {\large Dr. Maikol Solís Chacón}

    \vfill

    {\large \today\par}
\end{titlepage}
\section*{Resumen}
El análisis de fenómenos naturales extremos requiere metodologías que permitan capturar 
tanto la variabilidad individual de las variables involucradas como la compleja 
estructura de dependencia que las vincula. En este estudio se desarrolla un enfoque 
probabilístico multivariado orientado a la evaluación del riesgo asociado a desastres 
naturales en Costa Rica, considerando eventos como inundaciones, deslizamientos y sismos. 
El proceso inicia con un análisis exploratorio exhaustivo que permite caracterizar la 
estructura de los datos, identificar patrones relevantes y garantizar su calidad antes 
del modelado estadístico.

Posteriormente, se implementa el método de simulación Monte Carlo para generar un 
conjunto amplio de escenarios estocásticos que representan la variabilidad potencial de 
los fenómenos estudiados. Esto posibilita la estimación de distribuciones empíricas y 
métricas de riesgo como el Valor en Riesgo (VaR) a distintos niveles de confianza. Para 
capturar adecuadamente la dependencia entre variables, se ajustan diferentes modelos de 
cópulas, lo que permite representar relaciones no lineales y comportamientos de cola 
que los métodos tradicionales no logran reflejar.

Adicionalmente, se emplean transformaciones empíricas y técnicas vinculadas a la teoría 
de valores extremos, lo que facilita el análisis diferenciado por provincia y categoría, 
así como la construcción de métricas de riesgo específicas para cada región. Finalmente, 
se elaboran visualizaciones y estructuras comparativas que permiten identificar patrones 
espaciales y sectoriales de riesgo.

En conjunto, la metodología desarrollada proporciona una evaluación integral y precisa 
de los fenómenos extremos, ofreciendo una base sólida para la toma de decisiones en 
gestión del riesgo y planificación ante eventos catastróficos.

\newpage
\section*{Introducción}

El estudio de fenómenos extremos asociados a catástrofes naturales, tales como 
inundaciones, deslizamientos o sismos, ha experimentado un notable desarrollo gracias 
a la incorporación de enfoques probabilísticos multivariados capaces de representar con 
mayor precisión la compleja interacción entre variables físicas clave. Dichos enfoques 
permiten abordar la incertidumbre inherente a los eventos extremos mediante técnicas que 
separan el análisis marginal del estudio de la dependencia, ofreciendo así una estructura 
metodológica más flexible y coherente. En este marco, las cópulas se han consolidado 
como una herramienta fundamental, pues permiten construir distribuciones conjuntas a 
partir de distribuciones marginales arbitrarias, capturando dependencias no lineales y 
comportamientos en cola que los métodos tradicionales basados exclusivamente en 
correlación no logran reflejar adecuadamente \cite{DelfinerGutierrez2025,PerezGarcia2004}.

Paralelamente, la teoría de valores extremos (Extreme Value Theory, EVT) constituye un 
pilar esencial para el modelado de eventos raros y de alto impacto, proporcionando 
herramientas rigurosas para caracterizar el comportamiento de las colas de distribución 
y estimar la probabilidad de ocurrencia de fenómenos extraordinarios \cite{Siddiqui2022}. La combinación de EVT con modelos de dependencia basados en cópulas amplía significativamente la capacidad analítica, al permitir la evaluación conjunta de múltiples variables extremas, como la intensidad y duración de precipitaciones, o la simultaneidad de aceleraciones sísmicas en diversas direcciones estructurales.

Diversos estudios han demostrado la utilidad de esta integración metodológica en áreas 
como la hidrología y la ingeniería sísmica. Por ejemplo, en el análisis de riesgos de 
inundación, el empleo de vine copulas ha permitido modelar de manera precisa la 
dependencia entre la duración, el volumen y el pico de una crecida, obteniéndose 
probabilidades conjuntas más realistas y mejor fundamentadas que las que se derivarían 
de un análisis univariado \cite{DelfinerGutierrez2025}. De igual forma, en ingeniería 
estructural, la evaluación probabilística de la demanda sísmica basada en cópulas ha 
demostrado reducir la sobreestimación del riesgo que típicamente surge cuando se ignora 
la dependencia entre aceleraciones espectrales en distintos modos de vibración \cite{Siddiqui2022}.

El ámbito asegurador también ha incorporado estas metodologías para mejorar la cuantificación de pérdidas asociadas a catástrofes naturales. La integración de cópulas con técnicas de simulación, particularmente con el método de Monte Carlo, ha permitido caracterizar escenarios de pérdidas extremas y estimar métricas de riesgo como el Valor en Riesgo (VaR), proporcionando una base más sólida para el establecimiento de primas y para la gestión estratégica del riesgo financiero \cite{PerezGarcia2004}.

En consonancia con estas aplicaciones, el presente trabajo desarrolla un marco probabilístico integral orientado al estudio de desastres naturales en Costa Rica. Para ello, se emplea un enfoque estructurado que inicia con un análisis exploratorio detallado de los datos, continúa con la generación de escenarios mediante simulación Monte Carlo, incorpora posteriormente el modelado de dependencias a través de cópulas, y se complementa con la aplicación de técnicas de EVT y estimación del VaR a diferentes niveles de confianza. Asimismo, se realiza un análisis diferenciado por provincia y categoría utilizando transformaciones empíricas que permiten estudiar con mayor precisión la variación espacial y sectorial del riesgo.

En conjunto, este enfoque metodológico robusto permite capturar la complejidad inherente a los fenómenos naturales extremos, mejorar la representación de sus dependencias estadísticas y proporcionar estimaciones más realistas de los riesgos multivariados. Al integrar técnicas avanzadas de simulación, estadística multivariada y análisis de valores extremos, el presente estudio contribuye a fortalecer la capacidad de evaluación, prevención y toma de decisiones ante escenarios catastróficos en el contexto costarricense \cite{DelfinerGutierrez2025,Siddiqui2022,PerezGarcia2004}.
\newpage
\section*{Marco Teórico}


Los \textit{desastres naturales} son fenómenos que, al interactuar con condiciones de vulnerabilidad social y económica, producen efectos adversos en la población, la infraestructura y el entorno \cite{Paniagua1995}. Estos eventos no deben entenderse únicamente como expresiones de la naturaleza, sino también como hechos sociales, en tanto ponen de manifiesto desigualdades, falta de planificación y limitaciones en la capacidad de respuesta de las comunidades \cite{PerezMallaina2005}. En el caso de América Central y particularmente Costa Rica, estudios recientes han documentado la recurrencia de eventos sísmicos, inundaciones y erupciones volcánicas, señalando que estos han generado tanto desplazamientos poblacionales como impactos fiscales de consideración \cite{CentenoMorales2017,OrozcoMontoya2022}.

La gestión del riesgo financiero asociado a desastres naturales se ha abordado en el plano internacional con el Marco de Sendai para la Reducción del Riesgo de Desastres, cuyo propósito central es minimizar las pérdidas económicas y materiales derivadas de fenómenos naturales \cite{undrr2015}. Este marco global constituye una referencia fundamental, pues no se limita a proponer estrategias de respuesta, sino que impulsa la incorporación del riesgo en la planificación del desarrollo, tanto económico como social. En consecuencia, se reconoce que la reducción de pérdidas es también asunto de sostenibilidad fiscal y estabilidad macroeconómica. En el contexto de Costa Rica, la Comisión Nacional de Prevención de Riesgos y Atención de Emergencias ha planteado el Plan Nacional de Gestión del Riesgo 2021–2025, el cual establece mecanismos que buscan cuantificar y enfrentar las consecuencias económicas de los desastres \cite{cne2022}. De esta manera, se observa una transición de un modelo centrado en atender emergencias después de ocurridas, hacia un enfoque preventivo.

Las catástrofes naturales tienen efectos inmediatos y de largo plazo en la economía. Los costos directos incluyen daños a infraestructura, viviendas y sistemas de transporte, mientras que los indirectos abarcan la pérdida de productividad, la interrupción de cadenas de suministro y el aumento en gastos sociales y de salud \cite{Paniagua1995}. En contextos de alta vulnerabilidad, estos gastos pueden comprometer seriamente la sostenibilidad financiera de los Estados. En América Latina y el Caribe, organismos internacionales como el Banco Mundial han advertido sobre la brecha existente entre los costos económicos de los desastres y la capacidad de respuesta fiscal, lo cual genera presión sobre los presupuestos públicos y limita el margen de acción para la inversión en desarrollo \cite[p. 23]{bancamundial2021}. Consecuentemente, la Comisión Económica para América Latina y el Caribe documentó que entre 1990 y 2012 las pérdidas económicas anuales de la región superaron en promedio el 1\% del PIB, cifra que ilustra no solo la magnitud del impacto financiero, sino también la vulnerabilidad estructural de los países ante eventos extremos \cite[p. 45]{cepal2014}.

Estos fenómenos no solo afectan a las comunidades vulnerables, sino que también generan perturbaciones en sectores estratégicos como la agricultura, la energía y el turismo. Además, las proyecciones climáticas resaltan la necesidad de herramientas predictivas más robustas y de esquemas de financiamiento adaptativo que permitan a las instituciones responder de forma oportuna. En este sentido, el análisis del riesgo asociado a los desastres naturales se ha beneficiado de herramientas estadísticas avanzadas. Entre ellas destacan las \textit{cópulas}, que permiten modelar dependencias entre variables extremas, lo cual resulta especialmente útil cuando se analizan fenómenos multivariados como lluvias intensas, caudales fluviales o movimientos sísmicos, donde la correlación tradicional no resulta suficiente para capturar la complejidad de la relación \cite{patton2012review, krupskii2013factor}. En el ámbito regional, \citeA{RiveraVargas2022} aplicaron cópulas en un análisis probabilístico de peligro sísmico, mostrando su utilidad para evaluar riesgos conjuntos y calcular pérdidas esperadas en escenarios de alta dependencia. Asimismo, \citeA{Chand2024} demostraron que el uso de cópulas en la evaluación de riesgo de inundaciones permite una estimación más precisa de los gastos potenciales en infraestructura y mitigación.

\textbf{Desde una perspectiva actuarial y estadística}, los métodos empleados en este trabajo se sustentan en los fundamentos expuestos por \citeA{Klugman2019} en su obra \textit{Loss Models: From Data to Decisions}. Los autores destacan que la modelización de pérdidas requiere comprender tanto la frecuencia como la severidad de los eventos, enfatizando que “\textit{el comportamiento de la cola de la distribución juega un papel crítico en la cuantificación de los eventos extremos}” \cite[p. 412]{Klugman2019}. Esta afirmación respalda el uso de la teoría del valor extremo (EVT) como herramienta para estimar con precisión la probabilidad de pérdidas catastróficas. Según los mismos autores, “\textit{la aproximación por valores extremos proporciona una base teórica sólida para extrapolar más allá de los datos observados, permitiendo inferencias sobre eventos raros pero de gran impacto}” \cite[p. 415]{Klugman2019}.

La \textit{teoría del valor extremo} (EVT, por sus siglas en inglés) constituye así una herramienta fundamental para la modelación de desastres, ya que se centra en los eventos poco frecuentes pero de gran magnitud, precisamente los que generan los mayores impactos económicos \cite{Siddiqui2022}. Estudios pioneros aplicaron EVT al ajuste y modelación de catástrofes con el fin de estimar pérdidas máximas y diseñar mecanismos de financiamiento adecuados \cite{PerezGarcia2004}. Investigaciones más recientes amplían esta perspectiva al contexto financiero y asegurador, subrayando su valor para la gestión del riesgo catastrófico \cite{DelfinerGutierrez2025}. En términos de modelación, Klugman et al. afirman que “\textit{las distribuciones con colas pesadas, como Pareto o lognormal, son esenciales para reflejar adecuadamente la naturaleza de los riesgos extremos, donde pocas observaciones pueden dominar la media o la varianza}” \cite[p. 260]{Klugman2019}. Esta característica coincide con la lógica de la EVT, que busca estimar el comportamiento de las colas de distribución más allá del rango de observación empírica.

Asimismo, los métodos de cópulas empleados en este estudio también encuentran fundamento en la literatura actuarial moderna. Según \citeA{Klugman2019}, “\textit{las cópulas permiten construir modelos multivariados que capturan la dependencia entre riesgos sin requerir que las variables involucradas compartan la misma distribución marginal}” \cite[p. 489]{Klugman2019}. Esta propiedad resulta indispensable en el análisis de riesgos asociados a fenómenos naturales, donde las variables pueden presentar distribuciones heterogéneas y correlaciones no lineales. Por ello, la combinación de EVT y cópulas ofrece un marco robusto para estimar la probabilidad conjunta de pérdidas extremas, fortaleciendo la gestión del riesgo financiero y el cálculo de métricas como el \textit{Value at Risk (VaR)} y el \textit{Expected Shortfall (ES)}.

Los gastos tras un desastre natural suelen dividirse en atención inmediata, que incluye rescate, albergues temporales y distribución de víveres; reconstrucción, enfocada en la reparación de infraestructura, vivienda, sistemas de transporte y servicios públicos; y recuperación económica, mediante subsidios, reactivación de actividades productivas y apoyo a sectores estratégicos. La resiliencia económica de un país frente a desastres está determinada por la existencia de fondos de contingencia, seguros catastróficos y mecanismos de financiamiento internacional. La ausencia de estos mecanismos genera altos niveles de endeudamiento y limita la inversión futura en desarrollo \cite{CentenoMorales2017,OrozcoMontoya2022}.

En este contexto, la articulación entre el sector público, el sector privado y la academia desempeña un papel crucial en la creación de mecanismos financieros innovadores, como los seguros paramétricos y los bonos catastróficos. Estos instrumentos no solo distribuyen el riesgo, sino que también fortalecen la resiliencia financiera frente a eventos de gran magnitud \cite{quesada2020}. Sin embargo, aún persisten desafíos importantes en la coordinación de actores y en la generación de confianza, lo que limita la efectividad de estas herramientas. Además, el involucramiento activo del sector privado sigue siendo una tarea pendiente, especialmente en países en vías de desarrollo donde las estructuras de mercado no siempre favorecen la adopción de soluciones financieras de carácter preventivo.

En síntesis, el estudio de los desastres naturales y los gastos que estos implican requiere una aproximación interdisciplinaria que combine historia, ciencias sociales, economía, estadística avanzada y gestión pública. Este trabajo plantea la necesidad de integrar perspectivas tradicionales sobre vulnerabilidad y desigualdad con marcos de gobernanza internacional y modelos probabilísticos modernos, con el fin de diseñar estrategias de mitigación y financiamiento que reduzcan los impactos económicos y sociales de los desastres en la región y contribuyan a garantizar la estabilidad económica y social a largo plazo.


\section*{Metodología}

La metodología empleada en el presente estudio se basó en una combinación de técnicas 
estadísticas avanzadas orientadas al análisis de datos, la simulación estocástica y el 
modelado de dependencias. El proceso inició con un análisis exploratorio exhaustivo de 
la base de datos con el propósito de comprender la estructura de la información, 
identificar patrones relevantes y evaluar la presencia de valores atípicos, 
inconsistencias o posibles problemas de calidad. Durante esta etapa se examinaron 
distribuciones marginales, se generaron representaciones gráficas como histogramas, 
diagramas de cajas y mapas de correlación, y se efectuaron las transformaciones y 
limpiezas necesarias para garantizar la integridad de los datos utilizados en las fases 
posteriores.

Posteriormente, se implementó el método de simulación Monte Carlo con el fin de modelar 
la variabilidad inherente a las variables de interés y generar escenarios alternativos 
que permitieran aproximar el comportamiento estocástico del sistema bajo estudio. Para 
ello, se ajustaron distribuciones a los datos observados y se efectuó un número elevado 
de simulaciones, lo que permitió obtener distribuciones empíricas de resultados, 
cuantificar la incertidumbre y estimar medidas de riesgo como el Valor en Riesgo (VaR) 
bajo diversos niveles de confianza.

Con el objetivo de capturar la estructura de dependencia entre las variables, se 
procedió al ajuste de diferentes modelos de cópulas, los cuales resultan especialmente 
adecuados para representar relaciones no lineales y comportamientos de cola que no 
pueden ser descritos mediante la correlación tradicional. Este procedimiento incluyó la 
selección de familias de cópulas apropiadas, la estimación de sus parámetros mediante 
métodos de máxima verosimilitud y el análisis de medidas de dependencia como el 
coeficiente $\tau$ de Kendall. A partir de estas cópulas ajustadas se generaron simulaciones 
conjuntas que preservan la estructura dependiente observada en los datos originales.

Finalmente, se aplicaron técnicas de Transformación de Variables Empíricas (TVE) para 
analizar de manera más precisa las dependencias entre categorías y provincias específicas, 
complementando así el modelado previo. Esta etapa permitió transformar las variables a 
una escala uniforme, calcular matrices de dependencia diferenciadas y estimar el Valor 
en Riesgo tanto al 95 \% como al 99 \% para cada combinación de provincia y categoría. 
Con base en estos resultados se elaboraron representaciones visuales como mapas de calor, 
que facilitaron la identificación de patrones espaciales y sectoriales en los niveles de 
riesgo. En conjunto, todas estas etapas proporcionaron un enfoque metodológico robusto y 
coherente, permitiendo analizar de forma integral el comportamiento de las variables 
bajo estudio y obtener conclusiones fundamentadas en técnicas cuantitativas de alta 
precisión.

\section*{Resultados}
%Aquí va la sección de resultados y graficos relevantes
\newpage
\section*{Anexos}
Link del Repositorio de GitHub : \url{https://github.com/CA0307-II-2025/grupo-2}

\newpage
\nocite{*}% Mostrar todas
\bibliography{referencias}
\end{document}
