% Options for packages loaded elsewhere
% Options for packages loaded elsewhere
\PassOptionsToPackage{unicode}{hyperref}
\PassOptionsToPackage{hyphens}{url}
\PassOptionsToPackage{dvipsnames,svgnames,x11names}{xcolor}
%
\documentclass[
  11pt,
]{article}
\usepackage{xcolor}
\usepackage[margin=2cm]{geometry}
\usepackage{amsmath,amssymb}
\setcounter{secnumdepth}{5}
\usepackage{iftex}
\ifPDFTeX
  \usepackage[T1]{fontenc}
  \usepackage[utf8]{inputenc}
  \usepackage{textcomp} % provide euro and other symbols
\else % if luatex or xetex
  \usepackage{unicode-math} % this also loads fontspec
  \defaultfontfeatures{Scale=MatchLowercase}
  \defaultfontfeatures[\rmfamily]{Ligatures=TeX,Scale=1}
\fi
\usepackage{lmodern}
\ifPDFTeX\else
  % xetex/luatex font selection
  \setmainfont[]{Times New Roman}
\fi
% Use upquote if available, for straight quotes in verbatim environments
\IfFileExists{upquote.sty}{\usepackage{upquote}}{}
\IfFileExists{microtype.sty}{% use microtype if available
  \usepackage[]{microtype}
  \UseMicrotypeSet[protrusion]{basicmath} % disable protrusion for tt fonts
}{}
\makeatletter
\@ifundefined{KOMAClassName}{% if non-KOMA class
  \IfFileExists{parskip.sty}{%
    \usepackage{parskip}
  }{% else
    \setlength{\parindent}{0pt}
    \setlength{\parskip}{6pt plus 2pt minus 1pt}}
}{% if KOMA class
  \KOMAoptions{parskip=half}}
\makeatother
% Make \paragraph and \subparagraph free-standing
\makeatletter
\ifx\paragraph\undefined\else
  \let\oldparagraph\paragraph
  \renewcommand{\paragraph}{
    \@ifstar
      \xxxParagraphStar
      \xxxParagraphNoStar
  }
  \newcommand{\xxxParagraphStar}[1]{\oldparagraph*{#1}\mbox{}}
  \newcommand{\xxxParagraphNoStar}[1]{\oldparagraph{#1}\mbox{}}
\fi
\ifx\subparagraph\undefined\else
  \let\oldsubparagraph\subparagraph
  \renewcommand{\subparagraph}{
    \@ifstar
      \xxxSubParagraphStar
      \xxxSubParagraphNoStar
  }
  \newcommand{\xxxSubParagraphStar}[1]{\oldsubparagraph*{#1}\mbox{}}
  \newcommand{\xxxSubParagraphNoStar}[1]{\oldsubparagraph{#1}\mbox{}}
\fi
\makeatother


\usepackage{longtable,booktabs,array}
\usepackage{calc} % for calculating minipage widths
% Correct order of tables after \paragraph or \subparagraph
\usepackage{etoolbox}
\makeatletter
\patchcmd\longtable{\par}{\if@noskipsec\mbox{}\fi\par}{}{}
\makeatother
% Allow footnotes in longtable head/foot
\IfFileExists{footnotehyper.sty}{\usepackage{footnotehyper}}{\usepackage{footnote}}
\makesavenoteenv{longtable}
\usepackage{graphicx}
\makeatletter
\newsavebox\pandoc@box
\newcommand*\pandocbounded[1]{% scales image to fit in text height/width
  \sbox\pandoc@box{#1}%
  \Gscale@div\@tempa{\textheight}{\dimexpr\ht\pandoc@box+\dp\pandoc@box\relax}%
  \Gscale@div\@tempb{\linewidth}{\wd\pandoc@box}%
  \ifdim\@tempb\p@<\@tempa\p@\let\@tempa\@tempb\fi% select the smaller of both
  \ifdim\@tempa\p@<\p@\scalebox{\@tempa}{\usebox\pandoc@box}%
  \else\usebox{\pandoc@box}%
  \fi%
}
% Set default figure placement to htbp
\def\fps@figure{htbp}
\makeatother





\setlength{\emergencystretch}{3em} % prevent overfull lines

\providecommand{\tightlist}{%
  \setlength{\itemsep}{0pt}\setlength{\parskip}{0pt}}






\usepackage{booktabs}
\usepackage{longtable}
\usepackage{enumitem}
\usepackage{hyperref}
\hypersetup{colorlinks=true, linkcolor=blue, urlcolor=blue}
\usepackage{fontspec}
\usepackage{emoji}
\makeatletter
\@ifpackageloaded{caption}{}{\usepackage{caption}}
\AtBeginDocument{%
\ifdefined\contentsname
  \renewcommand*\contentsname{Table of contents}
\else
  \newcommand\contentsname{Table of contents}
\fi
\ifdefined\listfigurename
  \renewcommand*\listfigurename{List of Figures}
\else
  \newcommand\listfigurename{List of Figures}
\fi
\ifdefined\listtablename
  \renewcommand*\listtablename{List of Tables}
\else
  \newcommand\listtablename{List of Tables}
\fi
\ifdefined\figurename
  \renewcommand*\figurename{Figure}
\else
  \newcommand\figurename{Figure}
\fi
\ifdefined\tablename
  \renewcommand*\tablename{Table}
\else
  \newcommand\tablename{Table}
\fi
}
\@ifpackageloaded{float}{}{\usepackage{float}}
\floatstyle{ruled}
\@ifundefined{c@chapter}{\newfloat{codelisting}{h}{lop}}{\newfloat{codelisting}{h}{lop}[chapter]}
\floatname{codelisting}{Listing}
\newcommand*\listoflistings{\listof{codelisting}{List of Listings}}
\makeatother
\makeatletter
\makeatother
\makeatletter
\@ifpackageloaded{caption}{}{\usepackage{caption}}
\@ifpackageloaded{subcaption}{}{\usepackage{subcaption}}
\makeatother
\usepackage{bookmark}
\IfFileExists{xurl.sty}{\usepackage{xurl}}{} % add URL line breaks if available
\urlstyle{same}
\hypersetup{
  pdftitle={Planificación del Sprint},
  pdfauthor={Equipo de Proyecto},
  colorlinks=true,
  linkcolor={blue},
  filecolor={Maroon},
  citecolor={Blue},
  urlcolor={Blue},
  pdfcreator={LaTeX via pandoc}}


\title{Planificación del Sprint}
\author{Equipo de Proyecto}
\date{2025-09-05}
\begin{document}
\maketitle

\renewcommand*\contentsname{Table of contents}
{
\hypersetup{linkcolor=}
\setcounter{tocdepth}{3}
\tableofcontents
}

\section{📆 Planificación}\label{planificaciuxf3n}

\subsection{🎯 Objetivo del Sprint:}\label{objetivo-del-sprint}

\emph{Tener completado el marco teórico del proyecto y un análisis
exploratorio inicial de los datos.}

\subsection{😃 Historias de usuario}\label{historias-de-usuario}

\begin{itemize}
\tightlist
\item
  HU1 -- ``Desarrollo del marco teórico'' (Estimación: 5 pts) --
  \emph{Criterios de aceptación confirmados.}
\item
  HU2 -- ``Realizar análisis exploratorio de los datos'' (Estimación: 8
  pts) -- \emph{Criterios de aceptación confirmados.}
\end{itemize}

\subsection{🔜 Plan de alto nivel:}\label{plan-de-alto-nivel}

\begin{itemize}
\tightlist
\item
  \emph{Semana 1:} Redacción del marco teórico, búsqueda y organización
  de referencias.
\item
  \emph{Semana 2:} Limpieza preliminar de los datos y análisis
  exploratorio (tablas, gráficos, estadísticas descriptivas).
\end{itemize}

\subsection{🥇 Criterios de aceptación del
Sprint:}\label{criterios-de-aceptaciuxf3n-del-sprint}

\begin{itemize}
\tightlist
\item[$\boxtimes$]
  \emph{Marco teórico completo y revisado.}
\item[$\boxtimes$]
  \emph{Análisis exploratorio con visualizaciones y estadísticas
  descriptivas iniciales.}
\end{itemize}

\subsection{📌 Asignación de tareas
inicial}\label{asignaciuxf3n-de-tareas-inicial}

\begin{itemize}
\tightlist
\item
  \emph{Persona A:} Redacción inicial del marco teórico (HU1).
\item
  \emph{Persona B:} Revisión y pulido de bibliografía y citas (HU1).
\item
  \emph{Persona C:} Análisis exploratorio de datos (HU2).
\item
  \emph{Persona D:} Visualización de resultados y documentación (HU2).
\end{itemize}

\subsection{🚫 Posibles bloqueos o impedimentos
conocidos}\label{posibles-bloqueos-o-impedimentos-conocidos}

\begin{itemize}
\tightlist
\item
  \textbf{Bloqueo:} No había claridad inicial en los criterios para
  definir ``análisis exploratorio suficiente''.
\item
  \textbf{Solución:} Acordar como equipo incluir histogramas, boxplots y
  medidas estadísticas básicas como parte mínima de HU2.
\end{itemize}

\begin{center}\rule{0.5\linewidth}{0.5pt}\end{center}

\section{⏳ Daily}\label{daily}

\subsection{Fecha: 2025-09-01}\label{fecha-2025-09-01}

\subsubsection{Estudiante A:}\label{estudiante-a}

\begin{itemize}
\tightlist
\item
  \textbf{¿Qué hice ayer?}: Escribí el marco teórico sobre la
  importancia de los modelos estadísticos.
\item
  \textbf{¿Qué haré hoy?}: Integrar las referencias en formato correcto
  y pasarlas a .bib.
\item
  \textbf{¿Hay algo que me está bloqueando?}: Dificultad con el estilo
  de citación en LaTeX.
\end{itemize}

\subsubsection{Estudiante B:}\label{estudiante-b}

\begin{itemize}
\tightlist
\item
  \textbf{¿Qué hice ayer?}: Revisé la redacción del marco teórico y
  corregí coherencia.
\item
  \textbf{¿Qué haré hoy?}: Ajustar redacción final y entregar a
  revisión.
\item
  \textbf{¿Hay algo que me está bloqueando?}: No.
\end{itemize}

\subsubsection{Estudiante C:}\label{estudiante-c}

\begin{itemize}
\tightlist
\item
  \textbf{¿Qué hice ayer?}: Generé gráficos de dispersión y boxplots de
  las variables principales.
\item
  \textbf{¿Qué haré hoy?}: Calcular estadísticas descriptivas y
  correlaciones.
\item
  \textbf{¿Hay algo que me está bloqueando?}: No.
\end{itemize}

\subsubsection{Estudiante D:}\label{estudiante-d}

\begin{itemize}
\tightlist
\item
  \textbf{¿Qué hice ayer?}: Preparé tablas con medidas de tendencia
  central y variabilidad.
\item
  \textbf{¿Qué haré hoy?}: Organizar resultados y volcarlos en el
  documento.
\item
  \textbf{¿Hay algo que me está bloqueando?}: No.
\end{itemize}

\begin{center}\rule{0.5\linewidth}{0.5pt}\end{center}

\section{🔍 Revisión en clase (Fecha:
2025-09-05)}\label{revisiuxf3n-en-clase-fecha-2025-09-05}

\subsection{📈 Resultado mostrado}\label{resultado-mostrado}

\begin{itemize}
\tightlist
\item
  \emph{Funcionalidad A:} Marco teórico completo con referencias en
  formato APA.
\item
  \emph{Funcionalidad B:} Análisis exploratorio de datos con gráficos
  descriptivos y medidas.
\end{itemize}

\subsection{:arrows\_counterclockwise:
Retroalimentación}\label{arrows_counterclockwise-retroalimentaciuxf3n}

\begin{itemize}
\tightlist
\item
  \textbf{Profesor}: Sugirió agregar mayor detalle en la interpretación
  de gráficos.
\item
  \textbf{Compañeros:} Comentaron que el marco teórico quedó claro y
  bien estructurado.
\end{itemize}

\subsection{✔️ Criterios de aceptación
cumplidos:}\label{criterios-de-aceptaciuxf3n-cumplidos}

\begin{itemize}
\tightlist
\item[$\boxtimes$]
  HU1 -- Marco teórico.
\item[$\boxtimes$]
  HU2 -- Análisis exploratorio.
\end{itemize}

\begin{center}\rule{0.5\linewidth}{0.5pt}\end{center}

\section{🔙 Retrospective -- Fecha:
2025-09-05}\label{retrospective-fecha-2025-09-05}

\subsection{:white\_check\_mark: Qué salió
bien}\label{white_check_mark-quuxe9-saliuxf3-bien}

\begin{enumerate}
\def\labelenumi{\arabic{enumi}.}
\tightlist
\item
  Buena colaboración en redacción y revisión del marco teórico.
\item
  Se completó el análisis exploratorio con suficiente detalle.
\item
  La integración de resultados en el documento fue fluida.
\end{enumerate}

\subsection{:no\_good: Qué podría
mejorar}\label{no_good-quuxe9-podruxeda-mejorar}

\begin{itemize}
\tightlist
\item
  Aclarar con mayor anticipación los criterios de ``listo'' para HU
  técnicas.
\item
  Ajustar tiempos de trabajo en paralelo (hubo esperas entre HU1 y HU2).
\end{itemize}

\subsection{:pencil: Acciones concretas para el próximo
sprint}\label{pencil-acciones-concretas-para-el-pruxf3ximo-sprint}

\begin{enumerate}
\def\labelenumi{\arabic{enumi}.}
\tightlist
\item
  Definir criterios de aceptación más concretos para cada historia.
\item
  Planear tareas en paralelo para evitar dependencia entre equipo.
\item
  Mejorar el detalle en la interpretación de resultados exploratorios.
\end{enumerate}




\end{document}
